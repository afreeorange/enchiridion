% To do
% - Title page (headings in center) http://ctan.math.washington.edu/tex-archive/info/latex-samples/TitlePages/titlepages.pdf
% - headers on alternate pages
% - Image
%     https://en.wikipedia.org/wiki/Enchiridion_of_Epictetus#/media/File:Epictetus_Enchiridion_1683_page1.jpg
% - Margin for section numbers
% - The End
% - Different font for section numbers
% - Different paper size?

% \usepackage{fancyhdr}
% \pagestyle{fancy}
% \fancyhf{}
% \rhead{Share\LaTeX}
% \lhead{Guides and tutorials}
% \rfoot{Page \thepage}

\documentclass[a4paper,12pt]{book}

\usepackage{marginnote}
\usepackage{color}
\usepackage{fontspec}

% Set up the title page
\title{\uppercase{The Enchiridion}}
\author{\emph{Epictetus}}
\date{}

% Adjust paragraph indentation and spacing
\setlength{\parindent}{0em}
\setlength{\parskip}{1em}

% Set the document font
\defaultfontfeatures{Ligatures=TeX,Numbers=OldStyle}
\setmainfont{Adobe Caslon Pro}

% Create a 'ticker' for the original pages to be displayed in the margin
\newcounter{original_page_count}
\setcounter{original_page_count}{0}
\newcommand{\margincounter}{
    \stepcounter{original_page_count}
    \marginnote{
        \color{red}
        \small{
            \roman{original_page_count}
        }
    }[4mm]
}


\begin{document}
\maketitle

\margincounter{}

Some things are in our control and others not. Things in our control
are opinion, pursuit, desire, aversion, and, in a word, whatever are
our own actions. Things not in our control are body, property, reputation,
command, and, in one word, whatever are not our own actions.

The things in our control are by nature free, unrestrained, unhindered;
but those not in our control are weak, slavish, restrained, belonging
to others. Remember, then, that if you suppose that things which are
slavish by nature are also free, and that what belongs to others is
your own, then you will be hindered. You will lament, you will be
disturbed, and you will find fault both with gods and men. But if
you suppose that only to be your own which is your own, and what belongs
to others such as it really is, then no one will ever compel you or
restrain you. Further, you will find fault with no one or accuse no
one. You will do nothing against your will. No one will hurt you,
you will have no enemies, and you not be harmed. 

Aiming therefore at such great things, remember that you must not
allow yourself to be carried, even with a slight tendency, towards
the attainment of lesser things. Instead, you must entirely quit some
things and for the present postpone the rest. But if you would both
have these great things, along with power and riches, then you will
not gain even the latter, because you aim at the former too: but you
will absolutely fail of the former, by which alone happiness and freedom
are achieved. 

Work, therefore to be able to say to every harsh appearance, ``You
are but an appearance, and not absolutely the thing you appear to
be." And then examine it by those rules which you have, and first,
and chiefly, by this: whether it concerns the things which are in
our own control, or those which are not; and, if it concerns anything
not in our control, be prepared to say that it is nothing to you.

\margincounter{}

Remember that following desire promises the attainment of that
of which you are desirous; and aversion promises the avoiding that
to which you are averse. However, he who fails to obtain the object
of his desire is disappointed, and he who incurs the object of his
aversion wretched. If, then, you confine your aversion to those objects
only which are contrary to the natural use of your faculties, which
you have in your own control, you will never incur anything to which
you are averse. But if you are averse to sickness, or death, or poverty,
you will be wretched. Remove aversion, then, from all things that
are not in our control, and transfer it to things contrary to the
nature of what is in our control. But, for the present, totally suppress
desire: for, if you desire any of the things which are not in your
own control, you must necessarily be disappointed; and of those which
are, and which it would be laudable to desire, nothing is yet in your
possession. Use only the appropriate actions of pursuit and avoidance;
and even these lightly, and with gentleness and reservation.

\margincounter{}

With regard to whatever objects give you delight, are useful, or
are deeply loved, remember to tell yourself of what general nature
they are, beginning from the most insignificant things. If, for example,
you are fond of a specific ceramic cup, remind yourself that it is
only ceramic cups in general of which you are fond. Then, if it breaks,
you will not be disturbed. If you kiss your child, or your wife, say
that you only kiss things which are human, and thus you will not be
disturbed if either of them dies. 

\margincounter{}

When you are going about any action, remind yourself what nature
the action is. If you are going to bathe, picture to yourself the
things which usually happen in the bath: some people splash the water,
some push, some use abusive language, and others steal. Thus you will
more safely go about this action if you say to yourself, ``I will now
go bathe, and keep my own mind in a state conformable to nature."
And in the same manner with regard to every other action. For thus,
if any hindrance arises in bathing, you will have it ready to say,
``It was not only to bathe that I desired, but to keep my mind in a
state conformable to nature; and I will not keep it if I am bothered
at things that happen."

\margincounter{}

Men are disturbed, not by things, but by the principles and notions
which they form concerning things. Death, for instance, is not terrible,
else it would have appeared so to Socrates. But the terror consists
in our notion of death that it is terrible. When therefore we are
hindered, or disturbed, or grieved, let us never attribute it to others,
but to ourselves; that is, to our own principles. An uninstructed
person will lay the fault of his own bad condition upon others. Someone
just starting instruction will lay the fault on himself. Some who
is perfectly instructed will place blame neither on others nor on
himself. 

\margincounter{}

Don't be prideful with any excellence that is not your own. If
a horse should be prideful and say, ``I am handsome," it would be
supportable. But when you are prideful, and say, " I have a handsome
horse," know that you are proud of what is, in fact, only the good
of the horse. What, then, is your own? Only your reaction to the appearances
of things. Thus, when you behave conformably to nature in reaction
to how things appear, you will be proud with reason; for you will
take pride in some good of your own. 

\margincounter{}

Consider when, on a voyage, your ship is anchored; if you go on
shore to get water you may along the way amuse yourself with picking
up a shellish, or an onion. However, your thoughts and continual attention
ought to be bent towards the ship, waiting for the captain to call
on board; you must then immediately leave all these things, otherwise
you will be thrown into the ship, bound neck and feet like a sheep.
So it is with life. If, instead of an onion or a shellfish, you are
given a wife or child, that is fine. But if the captain calls, you
must run to the ship, leaving them, and regarding none of them. But
if you are old, never go far from the ship: lest, when you are called,
you should be unable to come in time. 

\margincounter{}

Don't demand that things happen as you wish, but wish that they
happen as they do happen, and you will go on well. 

\margincounter{}

Sickness is a hindrance to the body, but not to your ability to
choose, unless that is your choice. Lameness is a hindrance to the
leg, but not to your ability to choose. Say this to yourself with
regard to everything that happens, then you will see such obstacles
as hindrances to something else, but not to yourself. 

\margincounter{}

With every accident, ask yourself what abilities you have for
making a proper use of it. If you see an attractive person, you will
find that self-restraint is the ability you have against your desire.
If you are in pain, you will find fortitude. If you hear unpleasant
language, you will find patience. And thus habituated, the appearances
of things will not hurry you away along with them. 

\margincounter{}

Never say of anything, ``I have lost it"; but, ``I have returned
it." Is your child dead? It is returned. Is your wife dead? She is
returned. Is your estate taken away? Well, and is not that likewise
returned? ``But he who took it away is a bad man." What difference
is it to you who the giver assigns to take it back? While he gives
it to you to possess, take care of it; but don't view it as your own,
just as travelers view a hotel. 

\margincounter{}

If you want to improve, reject such reasonings as these: ``If I
neglect my affairs, I'll have no income; if I don't correct my servant,
he will be bad." For it is better to die with hunger, exempt from
grief and fear, than to live in affluence with perturbation; and it
is better your servant should be bad, than you unhappy. 

Begin therefore from little things. Is a little oil spilt? A little
wine stolen? Say to yourself, ``This is the price paid for apathy,
for tranquillity, and nothing is to be had for nothing." When you
call your servant, it is possible that he may not come; or, if he
does, he may not do what you want. But he is by no means of such importance
that it should be in his power to give you any disturbance.

\margincounter{}

If you want to improve, be content to be thought foolish and stupid
with regard to external things. Don't wish to be thought to know anything;
and even if you appear to be somebody important to others, distrust
yourself. For, it is difficult to both keep your faculty of choice
in a state conformable to nature, and at the same time acquire external
things. But while you are careful about the one, you must of necessity
neglect the other. 

\margincounter{}

If you wish your children, and your wife, and your friends to
live for ever, you are stupid; for you wish to be in control of things
which you cannot, you wish for things that belong to others to be
your own. So likewise, if you wish your servant to be without fault,
you are a fool; for you wish vice not to be vice," but something else.
But, if you wish to have your desires undisappointed, this is in your
own control. Exercise, therefore, what is in your control. He is the
master of every other person who is able to confer or remove whatever
that person wishes either to have or to avoid. Whoever, then, would
be free, let him wish nothing, let him decline nothing, which depends
on others else he must necessarily be a slave. 

\margincounter{}

Remember that you must behave in life as at a dinner party. Is
anything brought around to you? Put out your hand and take your share
with moderation. Does it pass by you? Don't stop it. Is it not yet
come? Don't stretch your desire towards it, but wait till it reaches
you. Do this with regard to children, to a wife, to public posts,
to riches, and you will eventually be a worthy partner of the feasts
of the gods. And if you don't even take the things which are set before
you, but are able even to reject them, then you will not only be a
partner at the feasts of the gods, but also of their empire. For,
by doing this, Diogenes, Heraclitus and others like them, deservedly
became, and were called, divine. 

\margincounter{}

When you see anyone weeping in grief because his son has gone
abroad, or is dead, or because he has suffered in his affairs, be
careful that the appearance may not misdirect you. Instead, distinguish
within your own mind, and be prepared to say, ``It's not the accident
that distresses this person., because it doesn't distress another
person; it is the judgment which he makes about it." As far as words
go, however, don't reduce yourself to his level, and certainly do
not moan with him. Do not moan inwardly either. 

\margincounter{}

Remember that you are an actor in a drama, of such a kind as the
author pleases to make it. If short, of a short one; if long, of a
long one. If it is his pleasure you should act a poor man, a cripple,
a governor, or a private person, see that you act it naturally. For
this is your business, to act well the character assigned you; to
choose it is another's. 

\margincounter{}

When a raven happens to croak unluckily, don't allow the appearance
hurry you away with it, but immediately make the distinction to yourself,
and say, ``None of these things are foretold to me; but either to my
paltry body, or property, or reputation, or children, or wife. But
to me all omens are lucky, if I will. For whichever of these things
happens, it is in my control to derive advantage from it."

\margincounter{}

You may be unconquerable, if you enter into no combat in which
it is not in your own control to conquer. When, therefore, you see
anyone eminent in honors, or power, or in high esteem on any other
account, take heed not to be hurried away with the appearance, and
to pronounce him happy; for, if the essence of good consists in things
in our own control, there will be no room for envy or emulation. But,
for your part, don't wish to be a general, or a senator, or a consul,
but to be free; and the only way to this is a contempt of things not
in our own control. 

\margincounter{}

Remember, that not he who gives ill language or a blow insults,
but the principle which represents these things as insulting. When,
therefore, anyone provokes you, be assured that it is your own opinion
which provokes you. Try, therefore, in the first place, not to be
hurried away with the appearance. For if you once gain time and respite,
you will more easily command yourself. 

\margincounter{}

Let death and exile, and all other things which appear terrible
be daily before your eyes, but chiefly death, and you win never entertain
any abject thought, nor too eagerly covet anything. 

\margincounter{}

If you have an earnest desire of attaining to philosophy, prepare
yourself from the very first to be laughed at, to be sneered by the
multitude, to hear them say, ``He is returned to us a philosopher
all at once," and ``Whence this supercilious look?" Now, for your
part, don't have a supercilious look indeed; but keep steadily to
those things which appear best to you as one appointed by God to this
station. For remember that, if you adhere to the same point, those
very persons who at first ridiculed will afterwards admire you. But
if you are conquered by them, you will incur a double ridicule.

\margincounter{}

If you ever happen to turn your attention to externals, so as
to wish to please anyone, be assured that you have ruined your scheme
of life. Be contented, then, in everything with being a philosopher;
and, if you wish to be thought so likewise by anyone, appear so to
yourself, and it will suffice you. 

\margincounter{}

Don't allow such considerations as these distress you. ``I will
live in dishonor, and be nobody anywhere." For, if dishonor is an
evil, you can no more be involved in any evil by the means of another,
than be engaged in anything base. Is it any business of yours, then,
to get power, or to be admitted to an entertainment? By no means.
How, then, after all, is this a dishonor? And how is it true that
you will be nobody anywhere, when you ought to be somebody in those
things only which are in your own control, in which you may be of
the greatest consequence? ``But my friends will be unassisted." --
What do you mean by unassisted? They will not have money from you,
nor will you make them Roman citizens. Who told you, then, that these
are among the things in our own control, and not the affair of others?
And who can give to another the things which he has not himself? ``Well,
but get them, then, that we too may have a share." If I can get them
with the preservation of my own honor and fidelity and greatness of
mind, show me the way and I will get them; but if you require me to
lose my own proper good that you may gain what is not good, consider
how inequitable and foolish you are. Besides, which would you rather
have, a sum of money, or a friend of fidelity and honor? Rather assist
me, then, to gain this character than require me to do those things
by which I may lose it. Well, but my country, say you, as far as depends
on me, will be unassisted. Here again, what assistance is this you
mean? ``It will not have porticoes nor baths of your providing." And
what signifies that? Why, neither does a smith provide it with shoes,
or a shoemaker with arms. It is enough if everyone fully performs
his own proper business. And were you to supply it with another citizen
of honor and fidelity, would not he be of use to it? Yes. Therefore
neither are you yourself useless to it. ``What place, then, say you,
will I hold in the state?" Whatever you can hold with the preservation
of your fidelity and honor. But if, by desiring to be useful to that,
you lose these, of what use can you be to your country when you are
become faithless and void of shame. 

\margincounter{}

Is anyone preferred before you at an entertainment, or in a compliment,
or in being admitted to a consultation? If these things are good,
you ought to be glad that he has gotten them; and if they are evil,
don't be grieved that you have not gotten them. And remember that
you cannot, without using the same means [which others do] to acquire
things not in our own control, expect to be thought worthy of an equal
share of them. For how can he who does not frequent the door of any
[great] man, does not attend him, does not praise him, have an equal
share with him who does? You are unjust, then, and insatiable, if
you are unwilling to pay the price for which these things are sold,
and would have them for nothing. For how much is lettuce sold? Fifty
cents, for instance. If another, then, paying fifty cents, takes the
lettuce, and you, not paying it, go without them, don't imagine that
he has gained any advantage over you. For as he has the lettuce, so
you have the fifty cents which you did not give. So, in the present
case, you have not been invited to such a person's entertainment,
because you have not paid him the price for which a supper is sold.
It is sold for praise; it is sold for attendance. Give him then the
value, if it is for your advantage. But if you would, at the same
time, not pay the one and yet receive the other, you are insatiable,
and a blockhead. Have you nothing, then, instead of the supper? Yes,
indeed, you have: the not praising him, whom you don't like to praise;
the not bearing with his behavior at coming in. 

\margincounter{}

The will of nature may be learned from those things in which we
don't distinguish from each other. For example, when our neighbor's
boy breaks a cup, or the like, we are presently ready to say, ``These
things will happen." Be assured, then, that when your own cup likewise
is broken, you ought to be affected just as when another's cup was
broken. Apply this in like manner to greater things. Is the child
or wife of another dead? There is no one who would not say, ``This
is a human accident." but if anyone's own child happens to die, it
is presently, ``Alas I how wretched am I!" But it should be remembered
how we are affected in hearing the same thing concerning others.

\margincounter{}

As a mark is not set up for the sake of missing the aim, so neither
does the nature of evil exist in the world. 

\margincounter{}

If a person gave your body to any stranger he met on his way,
you would certainly be angry. And do you feel no shame in handing
over your own mind to be confused and mystified by anyone who happens
to verbally attack you? 

\margincounter{}

In every affair consider what precedes and follows, and then undertake
it. Otherwise you will begin with spirit; but not having thought of
the consequences, when some of them appear you will shamefully desist.
``I would conquer at the Olympic games." But consider what precedes
and follows, and then, if it is for your advantage, engage in the
affair. You must conform to rules, submit to a diet, refrain from
dainties; exercise your body, whether you choose it or not, at a stated
hour, in heat and cold; you must drink no cold water, nor sometimes
even wine. In a word, you must give yourself up to your master, as
to a physician. Then, in the combat, you may be thrown into a ditch,
dislocate your arm, turn your ankle, swallow dust, be whipped, and,
after all, lose the victory. When you have evaluated all this, if
your inclination still holds, then go to war. Otherwise, take notice,
you will behave like children who sometimes play like wrestlers, sometimes
gladiators, sometimes blow a trumpet, and sometimes act a tragedy
when they have seen and admired these shows. Thus you too will be
at one time a wrestler, at another a gladiator, now a philosopher,
then an orator; but with your whole soul, nothing at all. Like an
ape, you mimic all you see, and one thing after another is sure to
please you, but is out of favor as soon as it becomes familiar. For
you have never entered upon anything considerately, nor after having
viewed the whole matter on all sides, or made any scrutiny into it,
but rashly, and with a cold inclination. Thus some, when they have
seen a philosopher and heard a man speaking like Euphrates (though,
indeed, who can speak like him?), have a mind to be philosophers too.
Consider first, man, what the matter is, and what your own nature
is able to bear. If you would be a wrestler, consider your shoulders,
your back, your thighs; for different persons are made for different
things. Do you think that you can act as you do, and be a philosopher?
That you can eat and drink, and be angry and discontented as you are
now? You must watch, you must labor, you must get the better of certain
appetites, must quit your acquaintance, be despised by your servant,
be laughed at by those you meet; come off worse than others in everything,
in magistracies, in honors, in courts of judicature. When you have
considered all these things round, approach, if you please; if, by
parting with them, you have a mind to purchase apathy, freedom, and
tranquillity. If not, don't come here; don't, like children, be one
while a philosopher, then a publican, then an orator, and then one
of Caesar's officers. These things are not consistent. You must be
one man, either good or bad. You must cultivate either your own ruling
faculty or externals, and apply yourself either to things within or
without you; that is, be either a philosopher, or one of the vulgar.

\margincounter{}

Duties are universally measured by relations. Is anyone a father?
If so, it is implied that the children should take care of him, submit
to him in everything, patiently listen to his reproaches, his correction.
But he is a bad father. Is you naturally entitled, then, to a good
father? No, only to a father. Is a brother unjust? Well, keep your
own situation towards him. Consider not what he does, but what you
are to do to keep your own faculty of choice in a state conformable
to nature. For another will not hurt you unless you please. You will
then be hurt when you think you are hurt. In this manner, therefore,
you will find, from the idea of a neighbor, a citizen, a general,
the corresponding duties if you accustom yourself to contemplate the
several relations. 

\margincounter{}

Be assured that the essential property of piety towards the gods
is to form right opinions concerning them, as existing ``I and as governing
the universe with goodness and justice. And fix yourself in this resolution,
to obey them, and yield to them, and willingly follow them in all
events, as produced by the most perfect understanding. For thus you
will never find fault with the gods, nor accuse them as neglecting
you. And it is not possible for this to be effected any other way
than by withdrawing yourself from things not in our own control, and
placing good or evil in those only which are. For if you suppose any
of the things not in our own control to be either good or evil, when
you are disappointed of what you wish, or incur what you would avoid,
you must necessarily find fault with and blame the authors. For every
animal is naturally formed to fly and abhor things that appear hurtful,
and the causes of them; and to pursue and admire those which appear
beneficial, and the causes of them. It is impractical, then, that
one who supposes himself to be hurt should be happy about the person
who, he thinks, hurts him, just as it is impossible to be happy about
the hurt itself. Hence, also, a father is reviled by a son, when he
does not impart to him the things which he takes to be good; and the
supposing empire to be a good made Polynices and Eteocles mutually
enemies. On this account the husbandman, the sailor, the merchant,
on this account those who lose wives and children, revile the gods.
For where interest is, there too is piety placed. So that, whoever
is careful to regulate his desires and aversions as he ought, is,
by the very same means, careful of piety likewise. But it is also
incumbent on everyone to offer libations and sacrifices and first
fruits, conformably to the customs of his country, with purity, and
not in a slovenly manner, nor negligently, nor sparingly, nor beyond
his ability. 

\margincounter{}

When you have recourse to divination, remember that you know not
what the event will be, and you come to learn it of the diviner; but
of what nature it is you know before you come, at least if you are
a philosopher. For if it is among the things not in our own control,
it can by no means be either good or evil. Don't, therefore, bring
either desire or aversion with you to the diviner (else you will approach
him trembling), but first acquire a distinct knowledge that every
event is indifferent and nothing to you., of whatever sort it may
be, for it will be in your power to make a right use of it, and this
no one can hinder; then come with confidence to the gods, as your
counselors, and afterwards, when any counsel is given you, remember
what counselors you have assumed, and whose advice you will neglect
if you disobey. Come to divination, as Socrates prescribed, in cases
of which the whole consideration relates to the event, and in which
no opportunities are afforded by reason, or any other art, to discover
the thing proposed to be learned. When, therefore, it is our duty
to share the danger of a friend or of our country, we ought not to
consult the oracle whether we will share it with them or not. For,
though the diviner should forewarn you that the victims are unfavorable,
this means no more than that either death or mutilation or exile is
portended. But we have reason within us, and it directs, even with
these hazards, to the greater diviner, the Pythian god, who cast out
of the temple the person who gave no assistance to his friend while
another was murdering him.

\margincounter{}

Immediately prescribe some character and form of conduce to yourself,
which you may keep both alone and in company. 

Be for the most part silent, or speak merely what is necessary, and
in few words. We may, however, enter, though sparingly, into discourse
sometimes when occasion calls for it, but not on any of the common
subjects, of gladiators, or horse races, or athletic champions, or
feasts, the vulgar topics of conversation; but principally not of
men, so as either to blame, or praise, or make comparisons. If you
are able, then, by your own conversation bring over that of your company
to proper subjects; but, if you happen to be taken among strangers,
be silent. 

Don't allow your laughter be much, nor on many occasions, nor profuse.

Avoid swearing, if possible, altogether; if not, as far as you are
able. 

Avoid public and vulgar entertainments; but, if ever an occasion calls
you to them, keep your attention upon the stretch, that you may not
imperceptibly slide into vulgar manners. For be assured that if a
person be ever so sound himself, yet, if his companion be infected,
he who converses with him will be infected likewise. 

Provide things relating to the body no further than mere use; as meat,
drink, clothing, house, family. But strike off and reject everything
relating to show and delicacy. 

As far as possible, before marriage, keep yourself pure from familiarities
with women, and, if you indulge them, let it be lawfully." But don't
therefore be troublesome and full of reproofs to those who use these
liberties, nor frequently boast that you yourself don't.

If anyone tells you that such a person speaks ill of you, don't make
excuses about what is said of you, but answer: " He does not know
my other faults, else he would not have mentioned only these."

It is not necessary for you to appear often at public spectacles;
but if ever there is a proper occasion for you to be there, don't
appear more solicitous for anyone than for yourself; that is, wish
things to be only just as they are, and him only to conquer who is
the conqueror, for thus you will meet with no hindrance. But abstain
entirely from declamations and derision and violent emotions. And
when you come away, don't discourse a great deal on what has passed,
and what does not contribute to your own amendment. For it would appear
by such discourse that you were immoderately struck with the show.

Go not [of your own accord] to the rehearsals of any 
(authors) , nor appear [at them] readily. But, if you do appear, keepyour
gravity and sedateness, and at the same time avoid being morose.

When you are going to confer with anyone, and particularly of those
in a superior station, represent to yourself how Socrates or Zeno
would behave in such a case, and you will not be at a loss to make
a proper use of whatever may occur. 

When you are going to any of the people in power, represent to yourself
that you will not find him at home; that you will not be admitted;
that the doors will not be opened to you; that he will take no notice
of you. If, with all this, it is your duty to go, bear what happens,
and never say [to yourself], ``It was not worth so much." For this
is vulgar, and like a man dazed by external things. 

In parties of conversation, avoid a frequent and excessive mention
of your own actions and dangers. For, however agreeable it may be
to yourself to mention the risks you have run, it is not equally agreeable
to others to hear your adventures. Avoid, likewise, an endeavor to
excite laughter. For this is a slippery point, which may throw you
into vulgar manners, and, besides, may be apt to lessen you in the
esteem of your acquaintance. Approaches to indecent discourse are
likewise dangerous. Whenever, therefore, anything of this sort happens,
if there be a proper opportunity, rebuke him who makes advances that
way; or, at least, by silence and blushing and a forbidding look,
show yourself to be displeased by such talk. 

\margincounter{}

If you are struck by the appearance of any promised pleasure,
guard yourself against being hurried away by it; but let the affair
wait your leisure, and procure yourself some delay. Then bring to
your mind both points of time: that in which you will enjoy the pleasure,
and that in which you will repent and reproach yourself after you
have enjoyed it; and set before you, in opposition to these, how you
will be glad and applaud yourself if you abstain. And even though
it should appear to you a seasonable gratification, take heed that
its enticing, and agreeable and attractive force may not subdue you;
but set in opposition to this how much better it is to be conscious
of having gained so great a victory. 

\margincounter{}

When you do anything from a clear judgment that it ought to be
done, never shun the being seen to do it, even though the world should
make a wrong supposition about it; for, if you don't act right, shun
the action itself; but, if you do, why are you afraid of those who
censure you wrongly? 

\margincounter{}

As the proposition, ``Either it is day or it is night," is extremely
proper for a disjunctive argument, but quite improper in a conjunctive
one, so, at a feast, to choose the largest share is very suitable
to the bodily appetite, but utterly inconsistent with the social spirit
of an entertainment. When you eat with another, then, remember not
only the value of those things which are set before you to the body,
but the value of that behavior which ought to be observed towards
the person who gives the entertainment. 

\margincounter{}

If you have assumed any character above your strength, you have
both made an ill figure in that and quitted one which you might have
supported. 

\margincounter{}

When walking, you are careful not to step on a nail or turn your
foot; so likewise be careful not to hurt the ruling faculty of your
mind. And, if we were to guard against this in every action, we should
undertake the action with the greater safety. 

\margincounter{}

The body is to everyone the measure of the possessions proper
for it, just as the foot is of the shoe. If, therefore, you stop at
this, you will keep the measure; but if you move beyond it, you must
necessarily be carried forward, as down a cliff; as in the case of
a shoe, if you go beyond its fitness to the foot, it comes first to
be gilded, then purple, and then studded with jewels. For to that
which once exceeds a due measure, there is no bound. 

\margincounter{}

Women from fourteen years old are flattered with the title of
``mistresses" by the men. Therefore, perceiving that they are regarded
only as qualified to give the men pleasure, they begin to adorn themselves,
and in that to place ill their hopes. We should, therefore, fix our
attention on making them sensible that they are valued for the appearance
of decent, modest and discreet behavior. 

\margincounter{}

It is a mark of want of genius to spend much time in things relating
to the body, as to be long in our exercises, in eating and drinking,
and in the discharge of other animal functions. These should be done
incidentally and slightly, and our whole attention be engaged in the
care of the understanding. 

\margincounter{}

When any person harms you, or speaks badly of you, remember that
he acts or speaks from a supposition of its being his duty. Now, it
is not possible that he should follow what appears right to you, but
what appears so to himself. Therefore, if he judges from a wrong appearance,
he is the person hurt, since he too is the person deceived. For if
anyone should suppose a true proposition to be false, the proposition
is not hurt, but he who is deceived about it. Setting out, then, from
these principles, you will meekly bear a person who reviles you, for
you will say upon every occasion, ``It seemed so to him."

\margincounter{}

Everything has two handles, the one by which it may be carried,
the other by which it cannot. If your brother acts unjustly, don't
lay hold on the action by the handle of his injustice, for by that
it cannot be carried; but by the opposite, that he is your brother,
that he was brought up with you; and thus you will lay hold on it,
as it is to be carried. 

\margincounter{}

These reasonings are unconnected: ``I am richer than you, therefore
I am better"; ``I am more eloquent than you, therefore I am better."
The connection is rather this: ``I am richer than you, therefore my
property is greater than yours;" ``I am more eloquent than you, therefore
my style is better than yours." But you, after all, are neither property
nor style. 

\margincounter{}

Does anyone bathe in a mighty little time? Don't say that he does
it ill, but in a mighty little time. Does anyone drink a great quantity
of wine? Don't say that he does ill, but that he drinks a great quantity.
For, unless you perfectly understand the principle from which anyone
acts, how should you know if he acts ill? Thus you will not run the
hazard of assenting to any appearances but such as you fully comprehend.

\margincounter{}

Never call yourself a philosopher, nor talk a great deal among
the unlearned about theorems, but act conformably to them. Thus, at
an entertainment, don't talk how persons ought to eat, but eat as
you ought. For remember that in this manner Socrates also universally
avoided all ostentation. And when persons came to him and desired
to be recommended by him to philosophers, he took and- recommended
them, so well did he bear being overlooked. So that if ever any talk
should happen among the unlearned concerning philosophic theorems,
be you, for the most part, silent. For there is great danger in immediately
throwing out what you have not digested. And, if anyone tells you
that you know nothing, and you are not nettled at it, then you may
be sure that you have begun your business. For sheep don't throw up
the grass to show the shepherds how much they have eaten; but, inwardly
digesting their food, they outwardly produce wool and milk. Thus,
therefore, do you likewise not show theorems to the unlearned, but
the actions produced by them after they have been digested.

\margincounter{}

When you have brought yourself to supply the necessities of your
body at a small price, don't pique yourself upon it; nor, if you drink
water, be saying upon every occasion, ``I drink water." But first consider
how much more sparing and patient of hardship the poor are than we.
But if at any time you would inure yourself by exercise to labor,
and bearing hard trials, do it for your own sake, and not for the
world; don't grasp statues, but, when you are violently thirsty, take
a little cold water in your mouth, and spurt it out and tell nobody.

\margincounter{}

The condition and characteristic of a vulgar person, is, that
he never expects either benefit or hurt from himself, but from externals.
The condition and characteristic of a philosopher is, that he expects
all hurt and benefit from himself. The marks of a proficient are,
that he censures no one, praises no one, blames no one, accuses no
one, says nothing concerning himself as being anybody, or knowing
anything: when he is, in any instance, hindered or restrained, he
accuses himself; and, if he is praised, he secretly laughs at the
person who praises him; and, if he is censured, he makes no defense.
But he goes about with the caution of sick or injured people, dreading
to move anything that is set right, before it is perfectly fixed.
He suppresses all desire in himself; he transfers his aversion to
those things only which thwart the proper use of our own faculty of
choice; the exertion of his active powers towards anything is very
gentle; if he appears stupid or ignorant, he does not care, and, in
a word, he watches himself as an enemy, and one in ambush.

\margincounter{}

When anyone shows himself overly confident in ability to understand
and interpret the works of Chrysippus, say to yourself, " Unless Chrysippus
had written obscurely, this person would have had no subject for his
vanity. But what do I desire? To understand nature and follow her.
I ask, then, who interprets her, and, finding Chrysippus does, I have
recourse to him. I don't understand his writings. I seek, therefore,
one to interpret them." So far there is nothing to value myself upon.
And when I find an interpreter, what remains is to make use of his
instructions. This alone is the valuable thing. But, if I admire nothing
but merely the interpretation, what do I become more than a grammarian
instead of a philosopher? Except, indeed, that instead of Homer I
interpret Chrysippus. When anyone, therefore, desires me to read Chrysippus
to him, I rather blush when I cannot show my actions agreeable and
consonant to his discourse. 

\margincounter{}

Whatever moral rules you have deliberately proposed to yourself.
abide by them as they were laws, and as if you would be guilty of
impiety by violating any of them. Don't regard what anyone says of
you, for this, after all, is no concern of yours. How long, then,
will you put off thinking yourself worthy of the highest improvements
and follow the distinctions of reason? You have received the philosophical
theorems, with which you ought to be familiar, and you have been familiar
with them. What other master, then, do you wait for, to throw upon
that the delay of reforming yourself? You are no longer a boy, but
a grown man. If, therefore, you will be negligent and slothful, and
always add procrastination to procrastination, purpose to purpose,
and fix day after day in which you will attend to yourself, you will
insensibly continue without proficiency, and, living and dying, persevere
in being one of the vulgar. This instant, then, think yourself worthy
of living as a man grown up, and a proficient. Let whatever appears
to be the best be to you an inviolable law. And if any instance of
pain or pleasure, or glory or disgrace, is set before you, remember
that now is the combat, now the Olympiad comes on, nor can it be put
off. By once being defeated and giving way, proficiency is lost, or
by the contrary preserved. Thus Socrates became perfect, improving
himself by everything. attending to nothing but reason. And though
you are not yet a Socrates, you ought, however, to live as one desirous
of becoming a Socrates. 

\margincounter{}

The first and most necessary topic in philosophy is that of the
use of moral theorems, such as, ``We ought not to lie;" the second
is that of demonstrations, such as, ``What is the origin of our obligation
not to lie;" the third gives strength and articulation to the other
two, such as, ``What is the origin of this is a demonstration." For
what is demonstration? What is consequence? What contradiction? What
truth? What falsehood? The third topic, then, is necessary on the
account of the second, and the second on the account of the first.
But the most necessary, and that whereon we ought to rest, is the
first. But we act just on the contrary. For we spend all our time
on the third topic, and employ all our diligence about that, and entirely
neglect the first. Therefore, at the same time that we lie, we are
immediately prepared to show how it is demonstrated that lying is
not right. 

\margincounter{}

Upon all occasions we ought to have these maxims ready at hand: 

\begin{quote}
``Conduct me, Jove, and you, O Destiny, 
Wherever your decrees have fixed my station." 
\par
\emph{Cleanthes}
\end{quote}

\begin{quote}
``I follow cheerfully; and, did I not, 
Wicked and wretched, I must follow still 
Whoever yields properly to Fate, is deemed 
Wise among men, and knows the laws of heaven."
\par
\emph{Euripides, Frag. 965} 
\end{quote}

And this third: 

\begin{quote}
``O Crito, if it thus pleases the gods, thus let it be. Anytus and Melitus 
may kill me indeed, but hurt me they cannot." 
\par
\emph{Plato's Crito and Apology} 
\end{quote}

\end{document} 
